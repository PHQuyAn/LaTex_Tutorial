\documentclass[12pt, a4paper]{report}

%Thư viện
\usepackage[utf8]{inputenc}
\usepackage[vietnamese]{babel}
\usepackage{amsmath, amsfonts, amsthm}
\usepackage{graphicx}
\usepackage{cite}
\usepackage{geometry}
\geometry{bottom=2cm, top=2cm, left=2cm, right=2cm}
\usepackage{hyperref}
\usepackage{float}

% Cấu hình hyperlink cho đẹp (bỏ khung đỏ)
\hypersetup{
    colorlinks=true,
    linkcolor=blue,
    filecolor=magenta,
    urlcolor=cyan,
    citecolor=red,
}

%--------Tiêu đề------------%
\title{\textbf{\huge NGHIÊN CỨU VÀ PHÁT TRIỂN MÔ HÌNH GAN TRONG TÁI TẠO ẢNH Y TẾ}}
\author{\textbf{Sinh viên thực hiện:}\\ Phạm Huỳnh Quý An}
\date{\today}


\begin{document}

% Hiện tiêu đề
\maketitle
\tableofcontents
\newpage

%------------Nội dung paper----------------%
\chapter{Cơ sở lý thuyết}

\section{Bài toán Super-Resolution}
Mục tiêu của bài toán là khôi phục ảnh độ phân giải cao $I^{HR}$ từ hình ảnh độ phân giải thấp $I^{LR}$. Mối quan hệ giữa chúng được mô hình hóa như sau:

\begin{equation} \label{eq:degradation}
    I^{LR} = \mathcal{D}(I^{HR};\delta)    
\end{equation}
Trong đó $\mathcal{D}$ là hàm giảm chất lượng (degradation function), và $\delta$ là tham số nhiễu.

\section{Hàm mất mát (Loss Functions)}
Để huấn luyện mạng Generator G, chúng ta kết hợp nhiều thành phần (Perceptual Loss).
\subsection{Content Loss}
Thay vì tính toán trên từng pixel, ta tính khoảng cách giữa các đặc trưng (feature maps) được trích xuất bởi mạng VGG19:
\begin{equation} \label{Content Loss}
    L_{content}=\frac{1}{W_{i,j}H_{i,j}} \sum_{x=1}^{W_{i,j}}\sum_{y=1}^{H_{i,j}}(\phi_{i,j}(I^{HR})_{x,y} - \phi_{x,y}(G(I^{LR}))_{x,y})^2
\end{equation}

\subsection{Adversarial Loss}
Thành phần này giúp ảnh sinh ra trông "thật" hơn, đánh lừa được Discriminator:
\begin{equation} \label{eq:Adversarial Loss}
    L_{adv}=\sum_{n=1}^{N}-\log D(G(I^{LR}))    
\end{equation}

\subsection{Tổng hợp hàm mục tiêu}
Hàm loss cuối cùng là tổng có trọng số:
\begin{equation} \label{eq:LossTotal}
    L_{total}=\lambda_1L_{content} + \lambda_2L_{adv} + \lambda_3L_{pixel}
\end{equation}

\chapter{Thực nghiệm và Kết quả}
\section{Độ đo đánh giá (Metrics)}
Chúng ta sử dụng chỉ số PSNR (Peak Signal-to-Noise Ratio) để đánh giá chất lượng ảnh tái tạo. Công thức như sau:
\begin{equation} \label{eq:PSNR}
    \text{PSNR}=10 \cdot \log_{10}(\frac{MAX_I^2}{MSE})
\end{equation}
Trong đó $MAX_I$ là giá trị pixel lớn nhất của ảnh (thường là 255).

\section{Kết quả so sánh}
Dưới đây là bảng so sánh hiệu năng giữa mô hình đề xuất (Our GAN) với các phương pháp truyền thống như Bicubic và SRGAN \cite{gupta2020super}

%_____________Bảng________________%
\begin{table}[H]
    \centering
    \caption{So sánh chỉ số PSNR và SSIM trên bộ dữ liệu X-ray}
    \label{tab:results} %đặt nhãn để tham chiếu
    \vspace{10pt}
    \begin{tabular}{|l|c|c|c|}
         \hline
         \textbf{Phương pháp} & \textbf{PSNR (dB)} $\uparrow$ & \textbf{SSIM} $\uparrow$ & \textbf{Thời gian (s)}\\
         \hline
         Bicubic Interpolation & 24.50 & 0.72 & 0.01 \\
         SRGAN (2017) & 28.15 & 0.85 & 0.15 \\
         ESRGAN (2018) & 29.40 & 0.88 & 0.22 \\
         \hline
         \textbf{Our Method (Ours)} & \textbf{30.12} & \textbf{0.91} & \textbf{0.18} \\
         \hline
    \end{tabular}
\end{table}

Như ta thấy ở Bảng \ref{tab:results}, mô hình của chúng tôi đạt chỉ số PSNR cao nhất là 30.12 dB, vượt trội hơn so với SRGAN \cite{gupta2020super}

\chapter{Kết luận}
Nghiên cứu này chứng minh tính hiệu quả của việc kết hợp Content Loss và Adversarial Loss

%-------Tài liệu kham khảo--------%
\renewcommand\bibname{Tài liệu kham khảo}
\bibliographystyle{ieeetr}
\bibliography{BaiTap4_Capstone/references}

\end{document}
















